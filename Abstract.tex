\documentclass[12pt, Arial]{article}
\title{Abstract}
\author{How to Teach a Class to Grade Itself Using Game Theory}
\date{}
\begin{document}
\maketitle
An efficient peer grading mechanism is proposed for grading the multitude of assignments in online courses. This novel approach is based on game theory and mechanism design. A set of assumptions and a mathematical model is ratified to simulate the dominant strategy behavior of students in a given mechanism. A benchmark function accounting for grade accuracy and workload is established to quantitatively compare effectiveness and scalability of various mechanisms. After iteration of mechanisms under increasingly realistic assumptions, three are proposed: Calibration, Improved Calibration, and Deduction. The Calibration mechanism performs as predicted by game theory when tested in an online crowdsourced experiment. The Deduction mechanism performs relatively well in the benchmark, outperforming traditional automated and peer grading systems. The mathematical model and benchmark opens the way for future derivative works to be performed and compared.
\end{document}
