\documentclass[12 pt]{article}

\usepackage[parfill]{parskip} 
\begin{document}
\section{The Problem}
In a given class of \emph{n} students and 1 professor, students must submit assignments, each of which have grades on a scale of 1 to 100. All of these assignments must be graded by someone, and no one can grade their own assignment. Our goal is to come up with a scheme that minimizes the maximum amount of work done by a single person, as well as the maximum deviation between any student's given grade and the grade their assignment deserves.

\subsection{Rules}
Grading one of the assignments costs 1 unit of effort, and will always find the correct score, \emph{i.e} the \emph{objective score}, and grading is the only way to find this objective score.
\subsection{Assumptions}
These are the assumptions that we have made in creating this model.

1) The objective score is from 0 to 100

2) Grading is binary, a grader can either grade or not grade, \emph{i.e} there is no way to use 0.5 units of effort

3) Everyone, including the professor, has the ability to get the correct objective score

4) Every assignment costs 1 unit of effort, no matter who grades or how difficult the assignment

5) Students want good grades

6) People don't like doing work

7) Students are selfish and don't care about their friends

8) Students all share a common happiness function, $H(g)$, where \emph{g} represents the grade the students received, and the output is their happiness. w.l.o.g. $H(0)=0$

9) Doing work costs one unit of happiness, so after grading a paper, a student's happiness can be expressed as $H(g)-W$,W is number of work units used

10) Everyone is risk neutral with respect to happiness

11) \emph{Students cannot communicate with each other}

\section{The First Model}
This model is quite unrealistic, but not by such a degree that it is unimaginable. We were, however, able to create a perfect solution if the problem took place in this situation.
\subsection{The Mechanism}
In our model, we have the professor grade one paper, called \emph{X}. The professor then distributes all the student's papers randomly among the students, while making sure that no student receives his or her own paper. Every student is also given a copy of \emph{X}.

The professor then tells the students that "Out of the two papers you were given, one of them has been graded by me, and the other hasn't. If you fail to give the correct grade for the one I graded, then you will receive a 0." The students then go about their business grading the papers, and individually report the grades they gave out to the professor. Then, just like the professor promised, if the graders graded the pre-graded, or \emph{calibrated} paper incorrectly, then that grader gets a 0 on the assignment. If they graded correctly, the get the maximum of {some minimum value, the assigned grade}.
\subsection{Why it works}
G is the maximum of {some minimum value, the assigned grade}.

The happiness of one student grading one paper is equivalent to $$\frac{H(G)+H(0)}{2}-1$$ Because the students have a 0.5 chance of getting a 0, and a one-half chance of getting G.

If they grade both papers, their happiness will be equal to $$H(G)-2$$Because they spent two units of energy, and are guaranteed to get G.

If they grade neither then the happiness will be $$H(0)$$ Because they automatically get a 0, but spend no energy.

This means that if $H(G) - 2 > H(0)$, then student will grade both papers. So, for the professor to guarantee that the students use effort and obtain the correct grades, he or she would choose a minimum value such that $H(G) > H(0) + 2$.
\subsection{The result}
The teacher expends one unit of effort, and the students each spend 2, as long as the teacher set the minimum value high enough.
Everything above $H^{-1}(2)$ will be graded correctly, while everything below it will receive a score of $H^{-1}(2)$.
This means that the maximum deviation is $H^{-1}(2)$, because this is the score that might be given to someone who deserves a 0. Expressed in happiness units, we get a value of 2.
The means that the value of the objective function in this case is 4, a very low value, which shows how effective this solution is.
\section{The Next Step}
We now have a model that works very well in an unrealistic scenario, so we decided that this solution was fine, and moved on to a new model. The goal for this models was to make it more realistic while still making it simple to understand, and have a simply solution that would be easy for students to understand. 

The major change we made in this model was one to assumption number 11. We changed it to:

\emph{All students are able to communicate with each other}

This provides a very different scenario than our original model, because now students will be able to figure out which papers are calibrated. However, we still wanted to try and generalize our previous solution and apply it to the new model, so we came up with the following scheme.
\subsection{The New Mechanism}
Our first attempt to apply a solution to this model was a near carbon copy of the previous scheme.
We arrange the class in a virtual circle, where every student grades the papers of the \emph{k} students to the right of them. The proffesor also grades every \emph{kth} paper starting from some random point in the circle. If a student misgrades the calibrated paper, they get a 0, otherwise they get the maximum of their objective score and the minimum value from our previous mechanism.
\subsection{Why it Works and Problems}
In this scheme, because of the overlapping papers, the students have no better way of figuring out which papers are calibrated than random guessing, even though they can all communicate with each other. As such, there are 3 options for the students.

1) They can grade 0 papers, giving them a happiness of $H(0) = 0$

2) They can grade all the papers they are assigned, giving them a happiness of $H(g) - k$

3) They can grade i papers giving them a happiness of $(k-i)\div k \cdot H(G)-i$. This means that it is always worse to grade i, because happiness from grading i is equal to i times happiness for grading 0 plus i-k times happiness for grading k.

The major problem with this method is the amount of work. The professor has to grade $\frac{n}{k}$ papers, which can be massive in a class of 1000+ students. For this reason we tried to come up with a different mechanism.
\section{A better mechanism}
In our new mechanism, we went with a completely different strategy, that we wanted to work better than our previous one. It works like this:

For each assignment, a new assignment is created to evaluate the performance of the graders. Every student receives 2 papers from different people, so that each paper overlaps with one other person, and every paper is distributed twice. The student then chooses how many points to subtract, and writes a justification for each one of these points. Graders are then given a contribution score from 1 to -1. From each of the 2 assignments they grade, they get their points deducted/total points deducted-0.5. Then, if the original writers of the papers can appeal their grade if they are unhappy with it. The professor will grade the paper, and if any of the graders were wrong, they get a 0. To compute the final score, the grader's final grade on the assignment is calculated as follows: 4 points times their contribution score are added or subtracted to H(your assigned grade). (Then your grade is converted from Happiness to an actual score). 
\subsection{Why it works}
The grader's have two choices, they can either grade the papers, or not grade them.

If the grader does not grade the paper, what score should he give it?

If they take off any points, the points will surely be refuted by the professor during the regrade.
So if they take off any points, they will get a 0 on the grading assignment.
If the grader's give it 100, they might get 50 if the other grader also gives it 100. If he gives it anything else, the original grader will get 0.
$$$$
If the grader does grade the paper, what score should he give it?

If he gives it a lower score, they throw away free points, and if they give it a lower score there is a risk of them getting a 0 because the scores could be refuted. So, giving it the correct score is the best because it guarantees at least a 50.

So assuming that your partner gives the paper a 100, which is the best thing they could do if they don't grade, then
	If you don�t grade the paper, you don�t use any effort. So your happiness will H(your grade on the original assignment). Contribution score = 0
	If you do grade, you use 1 unit of effort by default. But, your happiness will be H(your grade on the original assignment) + 3. Contribution score = 1
	
If your partner gives the paper its true grade:
	If you don�t grade the assignment, you will give it a 100 and so your contribution score is -1. So your happiness is H(your original grade) - 4.
	If you do grade the assignment, you and your partner gave the same grades, so get a contribution score of 0. This means that your happiness is H(your original grade). 
	
Therefore, it is in everyone's best interest to honestly grade the paper. 
\subsection{The Result}
Assuming that everyone acts in the dominant strategy behavior, the maximum work for teacher is 0, the maximum work done for students is 2. 
Because everyone honestly grades the papers, the maximum deviation is 0.
So, the objective function is 2, which is even better than our first mechanism.
\section{The Next Step}
\subsection{What is wrong?}
Although the result that we produced for our previous model is rather good, it encourages very harsh competition between students, and would make the classroom environment feel very negative. This means that we need to find a different solution to the same problem which encourages a kind of positive competition, instead of the negatives formed by trying to find the most mistakes
\subsection{The next mechanism}
For the next mechanism, we kept all of our assumptions the same, and simply tried to find a way to better the feeling of negative competition. To do this, we start by having each paper get graded by two people. Then, the original writer of this paper chooses which of the two grades he/she would like to have, and looks over the justifications for losing any points on both papers. If they see that any of these justifications are incorrect, they flag the paper and send it to the professor. Following this, the paper they chose to have is sent to a third party, who grades the paper, and checks if their grade is close to the one that was given to the paper. If it isn't, the paper is sent to the professor. After everything that needs to be flagged has been, the professor grades all the flagged papers, and rewards people who were correct with a 2 point bonus. If no papers from any group were flagged, every grader in that group gets two points. Final grades are whatever the person chose if there was no flagging, or whatever the professor gave them if there was.
\subsection{Does it work?}
To find out how efficient this mechanism is, we need to analyze two things:
1) Is everyone an honest flagger?
$$$$
2) Do the graders grade honestly
\subsubsection{Flagging}
There are two cases for this:

1) They guess a random score and try to flag everything

-In this case, the chance of gaining points is so low that its disregardable

2) They take the time to flag correctly

-In this case they could get 2 points back, so it is the most beneficial
\subsubsection{Grading}
Once again, there are two cases here: People grading, and people not grading

If people don't bother grading anything, what score should they give the paper?

-100 because anything lower than the actual score will be flagged by the writer, and because of useless justifications, they will lose points. 100 is the only score guaranteed to be greater than or equal to the correct score

Their happiness in this case will be 0, because they use no effort, but also have no chance of gaining points since their paper will definitely get flagged

The happiness of people who honestly grade, on the other hand, will be a +1 gain, because they use 1 unit of effort to grade and they get 2 units back guaranteed. This higher gain means that it is beneficial to grade honestly.

\subsection{The result}
Because everyone is honest, we can now analyze the score of the mechanism:

Maximum work done by one person: 3 (student who flags one paper and grades 2 others)

Maximum deviation: 1 (1 point gain for being a good grader)

Final score: 4

Although this is not as good as our previous mechanism, it is still an improvement because there is now positive competition instead of negaitve
\end{document}
